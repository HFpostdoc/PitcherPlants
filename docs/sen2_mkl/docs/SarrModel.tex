%% LyX 2.1.3 created this file.  For more info, see http://www.lyx.org/.
%% Do not edit unless you really know what you are doing.
\documentclass[english]{article}
\usepackage[T1]{fontenc}
\usepackage[latin9]{inputenc}
\usepackage[letterpaper]{geometry}
\geometry{verbose,tmargin=2.54cm,bmargin=2.54cm,lmargin=2.54cm,rmargin=2.54cm}
\setcounter{secnumdepth}{-2}
\setcounter{tocdepth}{-2}
\usepackage{color}
\usepackage{array}
\usepackage{amsmath}
\usepackage{setspace}
\usepackage[authoryear]{natbib}
\usepackage{subscript}
\doublespacing

\makeatletter

%%%%%%%%%%%%%%%%%%%%%%%%%%%%%% LyX specific LaTeX commands.
%% Because html converters don't know tabularnewline
\providecommand{\tabularnewline}{\\}

%%%%%%%%%%%%%%%%%%%%%%%%%%%%%% Textclass specific LaTeX commands.
\newcommand{\lyxaddress}[1]{
\par {\raggedright #1
\vspace{1.4em}
\noindent\par}
}

%%%%%%%%%%%%%%%%%%%%%%%%%%%%%% User specified LaTeX commands.
\usepackage{babel}
\usepackage{lineno}
\usepackage{amsmath}

\linenumbers

\makeatother

\usepackage{babel}
\begin{document}

\title{Regime shifts and alternative states in the \emph{Sarracenia} microecosystem}


\author{Aaron M. Ellison,\textsuperscript{1,{*}} Benjamin Baiser,\textsuperscript{1,2}
Matthew K. Lau,\textsuperscript{1} and Nicholas J. Gotelli\textsuperscript{3}}

\maketitle

\lyxaddress{\textsuperscript{1}Harvard Forest, Harvard University, 324 North
Main Street, Petersham, Massachusetts 01366; \textsuperscript{2}Department
of Wildlife Ecology and Conservation, University of Florida, Gainesville,
Florida 32611; \textsuperscript{3}Department of Biology, University
of Vermont, Burlington, Vermont 05405}

\emph{Keywords}: alternative states, model system, organic-matter
loading, regime shifts, \emph{Sarracenia purpurea}

Article type: \textbf{Article}

\pagebreak{}


\part{Abstract}

text - 200 words, 1 paragraph, 

\begin{itemize}
\item Sudden, rapid ecosystem changes are particularly
  troubling when they are also followed by slow return rates or shifts
  to entirely new states. Studies of such dynamics are hampered by
  spatial and temporal scales of most terrestrial and aquatic
  ecosystems. 
\item Here, we use the micro-ecosystem associated with the carnivorous
  eastern pitcher plant (\textit{Sarracenia purpurea}) to explore the
  factors that contribute to ecosystem hysteresis after strong
  perturbations from nutrient addition.
\item Using sensitivity analyses we identify model parameters that
  most strongly control the dynamics of the pitcher plant ecosystem
  and show that the system can eventually overcome its hysteresis and
  return to an oligotrophic state once organic-matter input is
  stopped.
\end{itemize}

\pagebreak{}


\part*{Introduction}

Regime shifts in ecological systems are defined as rapid changes in
the spatial or temporal dynamics of a more-or-less stable system caused
by slow, directional changes in one or more underlying state variables
\citep{Scheffer2001,Scheffer2009}. Ecological systems in which the
occurrence of regime shifts is unambiguous are uncommon, primarily
because long time series of observations are required to identify
both stability of each state and a breakpoint between them \citep{Bestelmeyer2011}.
Detailed modeling and decades of observations and experiments have
led to a thorough understanding of a canonical example of an ecological
regime shift: the rapid shift from oligotrophic to eutrophic states
in lakes \citep{Carpenter2006a,Carpenter2011}. 

Oxygen dynamics in lakes can be described using a simple model that
yields both alternative states and hysteresis in the shift between
them \citep{Scheffer2001}: 

\begin{equation}
\frac{dx}{dt}=a-bx+rf(x)\label{eq:0:MinimalModel}
\end{equation}
In this model, the observed variable \emph{x} (e.g., nutrient concentration
in a lake) is positively correlated with state variable \emph{a} (e.g.,
rate of nutrient loading) and negatively correlated with state variable
\emph{b} (e.g., rate of nutrient removal). A positive feedback loop,
\emph{rf}(\emph{x}) (in the lake model, this is the rate of nutrient
recycling between the lake sediments and water column). If \emph{r
}> 0 and $\{rf\prime(x)\}>b$, there will be more than one equilibrium
point (i.e., stable state); the function $f(x)$ determines the shape
of the switch between the states and the degree of hysteresis \citep{Scheffer2001}. 

Models of lake ecosystems and their food webs, and associated empirical
data also have revealed that returning lakes to the more ``desired''
oligotrophic state can be very slow\textemdash on the order of decades
to centuries\textemdash (ref: Contamin and Ellison; others from Carpenter
group) and depends not only on slowing or reversing the directional
changes in underlying state variables but also on the internal feedback
dynamics of the system (i.e., the dynamics of $f(x)$ in Equation
\ref{eq:0:MinimalModel}). Other systems, including fisheries (ref:
Carpenter PNAS) and ... (ref; XXX et al. PLoS One) have provided some
support for these model results (ref: XXX et al. PLoS One), both in
terms of dynamics and duration.

In the last several years, many researchers have suggested that a
wide range of ecological systems are poised to ``tip'' into new
regimes (refs...), or even that we are approaching a planetary tipping
point. (ref: XXX, but see YYY). If, as in lakes, either changes in
the underlying state variables causing these regime shifts or the
regime shifts themselves span decades, ecological insights into their
causes and consequences will accrue relatively slowly. More rapid
progress both in understanding the mechanisms driving ecological regime
shifts and in determining how to reverse them requires well-understood
model systems that can be experimentally manipulated over much shorter
time scales.

Recently, we have shown experimentally that organic-matter loading
(i.e., prey addition) can cause a shift from oligotrophic to eutrophic
conditions in naturally-occurring microecosystems: the water and bacteria-filled
modified leaves of the northern (or purple) pitcher plant, \emph{Sarracenia
purpurea} (ref: Sirota). Because bacteria that reproduce rapidly drive
the nutrient cycling dynamics of the \emph{Sarracenia} microecosystem
(ref: Butler et al. 2008), the system states change in days rather
than years or decades. In addition to introducing a new experimental
system for regime shifts and alternative states, Sirota et al. (2013)
sketched a model that, in extreme cases, could lead to a rapid (regime)
shift from oligotrophic to eutrophic conditions in pitcher-plant leaves.
Here, we develop fully the model of the \emph{Sarracenia} microecosystem,
introduce more realism into the underlying environmental drivers of
the model, use sensitivity analysis to identify the parameters that
most strongly control the dynamics of the system, and show that the
system eventually can overcome its hysteresis and return to an oligotrophic
state once organic-matter input is stopped. 


\part*{The \emph{Sarracenia} microecosystem}

Sketch of the biology of the system


\part*{The Model}

The detritus-based pitcher-plant food web forms in the small pools
of water that accumulate in living and photosynthesizing modified
leaves (``pitchers'') of \emph{Sarracenia purpurea}. Thus, our model
of this food web includes feedbacks between the food web and the plant
itself \citep{Bradshaw1984}. The full model is a pair of coupled
differential equations:\begin{subequations}

\begin{align}
x_{t+1}=\underbrace{a_{t}A_{t}}_{\mathrm{Photosynthesis}}-\underbrace{\left\{ m+a_{t}\left[\frac{w_{t-1}}{K_{w}+w_{t-1}}\right]\right\} }_{f(w,x,t)}+\underbrace{D_{t}(x_{t})}_{g(x,t)}\label{eq:1a:FullModel}\\
a_{t+1}=a_{t}\times\left\{ \frac{a{}_{max}^{\prime}-a{}_{min}^{\prime}}{1+e^{-s\cdot n_{t}-d}}+a{}_{min}^{\prime}\right\} \label{eq:1b:FullModelIterate}
\end{align}
\end{subequations}In this model, the pitcher-plant fluid is oxygenated
by photosynthesis (\emph{A\textsubscript{\emph{t}}}), with \emph{A\textsubscript{max}}
augmented by prey mineralization (\emph{a\textsubscript{\emph{t}}}),
as well as some diffusion from the atmosphere (\emph{g}(\emph{x,t})).
Oxygen is lost from the fluid as prey is decomposed by the food web
(\emph{f}(\emph{w,x,t})). A full list of model parameters is given
in Table \ref{tab:ModelTerms}.

\begin{table}
\begin{raggedright}
\protect\caption{\label{tab:ModelTerms}Terms, units, and interpretation for the model
of oxygen dynamics in pitcher-plant fluid}

\par\end{raggedright}

\centering{}%
\begin{tabular*}{0.75\paperwidth}{@{\extracolsep{\fill}}cc>{\raggedright}p{0.5\textwidth}}
\hline 
\noalign{\vskip\doublerulesep}
Term & Units & Interpretation\tabularnewline
\hline 
\noalign{\vskip\doublerulesep}

\emph{t} & minutes & time (model iteration)\tabularnewline
$x_{t}$ & mg/L & {[}O\textsubscript{2}{]}: concentration of oxygen in
the pitcher fluid \tabularnewline

\textbf{Photosynthesis} \tabularnewline
$A_{t}$ & mg/L & Augmentation of oxygen infused from the plant to the
fluid at mid-day\tabularnewline
$a_{t}$ & mg/L & Augmentation of oxygen infused from the plant to the
fluid at mid-day\tabularnewline
$a_{min}^{\prime},\,a_{max}^{\prime}$ & mg/L & minimum or maximum
possible augmentation of photosynthesis as a result of nutrient
fertilization from organic-matter mineralization by the food
web\tabularnewline

\textbf{Respiration} \tabularnewline
$w_{t}$ & mg & mass of prey remaining at time \emph{t}\tabularnewline
$K_{w}$ & mg/min & half-saturation constant for prey
consumption\tabularnewline \emph{m} & mg/L & amount of oxygen used for
basal metabolism (respiration) of bacteria\tabularnewline

\textbf{Nutrient Dynamics} \tabularnewline
$n_{t}$ & mg/L & quantity of nutrients mineralized by decomposition; a
function of $w_{t}$ and $x_{t}$\tabularnewline
\emph{s} & dimensionless & steepenss of the sigmoidal curve relating
nutrient mineralization to photosynthetic augmentation\tabularnewline
\emph{d} & mg & inflection point of the sigmoidal curve relating nutrient mineralization
to photosynthetic augmentation\tabularnewline

\textbf{Diffusion} \tabularnewline 
$D_{t}$ & mg/L & Diffusion rate of oxygen from atmosphere into the
pitcher fluid\tabularnewline f & 1/\emph{t} & constant adjusting sine
wave of diurnal PAR for frequency of measurements\tabularnewline

\hline 
\end{tabular*}
\end{table}



\section*{Photosynthesis}

In the absence of the foodweb, baseline oxygen concentration {[}O\textsubscript{2}{]}
in the pitcher water, is determined by diurnal photosynthesis: pitchers
take up CO\textsubscript{2} from the air or water and release O\textsubscript{2}
back into it \citep{Cameron1977,Bradshaw1984,Sirota2013}. Photosynthesis
is first modeled as $A=A_{max}[1-\exp(-A_{qe}[\mathrm{PAR-LCP}])]$
(after \citealt{Peek2002}), where PAR is photosynthetically active
radiation, which normally ranges from 0 - 2500 $\mu\mathrm{mol\,\cdot\,m^{-2}\,\cdot\,s^{-1}}$
and which is modeled as a truncated dirunal sine-wave (PAR = 0 at
night, a sine-wave during the day: \textbf{\textcolor{magenta}{\ref{fig:PAR and A}}});
\emph{A} is photosynthetic rate (in $\mu\mathrm{mol\:CO_{2}\,\cdot\,m^{-2}\,\cdot\,s^{-1}}$);
\emph{A}\textsubscript{max} is the maximum photosynthetic rate of
an pitcher plant that has not captured or processed any prey ($\bar{x}\approx4\,\mu\mathrm{mol\:CO_{2}\,\cdot\,m^{-2}\cdot\,s^{-1}}$;
\citealt{Small1972,Farnsworth2008}); \emph{A\textsubscript{\emph{qe}}}
is the quantum yield ($\approx0.3$: the initial slope, at low light,
of the \emph{A} vs. PAR curve); and LCP is the light compensation
point: the x-intercept when A = 0, $\approx20\,\mathrm{mol\,\cdot\,m^{-2}\,\cdot\,s^{-1}}$. 

Since we are interested in oxygen dynamics in volumes of water, we
convert A from $\mu\mathrm{mol\:CO_{2}\,\cdot\,m^{-2}\cdot\,s^{-1}}$
to mg O\textsubscript{2} / L by noting that there is a 1:1 relationship
between $\mu$mol CO\textsubscript{2} and $\mu$mol O\textsubscript{2}
in phothosynthesis; that 1 $\mu$mol O\textsubscript{2}= 32 $\mu\mathrm{g\,O_{2}\,\cdot\,m^{-2}\,\cdot\,s^{-1}}$;
and that the specific leaf area of \emph{S. purpurea }$\approx$ 70
cm\textsuperscript{2}/g \citep{Farnsworth2008}. Thus, O\textsubscript{2}
production $\approx$ 0.21 $\mu\mathrm{g\,O_{2}\,\cdot\,g\:leaf^{-1}\,\cdot\,s^{-1}}$.
The average 5-ml leaf has a surface area of 40 cm\textsuperscript{2}
and weighs 0.5 g, and a 1-L leaf would weigh 10g. Thus, O\textsubscript{2}
production from photosynthesis $\approx$ 2.1 $\mathrm{mg\,O_{2}\,\cdot\,L^{-1}\,\cdot\,s^{-1}}$
. 

It follows that our functions of PAR and \emph{A}, taking into account
LCP, are:

\begin{subequations}

\begin{equation}
\mathrm{PAR}=c\sin(2\pi ft)\label{eq:PAR as sinewave}
\end{equation}
\begin{equation}
A=\begin{cases}
a\,\cdot\,A_{max}[1-\exp(-0.3[\mathrm{PAR}-20])] & :\mathrm{PAR}\geq20\\
0 & :\mathrm{PAR}<20
\end{cases}\label{eq:O2 conditional}
\end{equation}


\end{subequations}

In Eq. \ref{eq:PAR as sinewave}, the amplitude \emph{c} is maximum
PAR at mid-day; the constant \emph{f} adusts for the frequency of
measurements. Since our model updates every minute, \emph{f} = 1/1440
(the reciprocal of the number of minutes per day). Without loss of
generality, our model runs start at sunrise (\emph{t} = 1), day-length
= 12 hours = 720 minutes, and PAR(t) is trunctated = 0 for $t\in\{721,\,1440\}$.
In Eq. \ref{eq:O2 conditional}, \emph{a} is the amount that photosynthetic
rate can be increased as a result of nutrients provided to the plant
as prey is mineralized by the food web (\textbf{\textcolor{magenta}{\ref{fig:PAR and A}}}).

\begin{figure}
\protect\caption{\label{fig:PAR and A}Four-part figure. (a) truncated sine wave of
PAR; (b) observed PAR; (c) A as a function of PAR w/o food web; (d)
A as a function of PAR with mineralization augmentation}



\end{figure}



\section*{Respiration}

Biological oxygen demand (\textsc{bod}) of bacteria as they decompose
prey rapidly depletes {[}O\textsubscript{2}{]} in the pitcher fluid.
Although different {[}O\textsubscript{2}{]} may favor different numbers
of either aerobic or anaerobic bacteria in the pitcher fluid, the
efficiency of organic-matter decomposition by either type of bacteria
is roughly equal \citep{Murphy1984}, so we assume sufficient numbers
of bacteria to decompose prey at a fixed, negative exponential rate:
\begin{equation}
w(t)=ae^{-b[w_{0}t]}\label{eq:decomp-negative exponential}
\end{equation}


Biologically, we take this to mean that easily digested parts of insect
prey, such as fat bodies, are processed quickly, whereas the more
recalcitrant proteins and chitins break down more slowly \citep{Baiser2011}.
Field observations have shown that the soft parts of a single $75-\mu g$
wasp can be comsumed completely over a 48-hour period in a pitcher
with 5 ml of fluid. Thus, the baseline parameters for Equation \ref{eq:decomp-negative exponential}
were set at \emph{a }= 20 mg and $b\,=\,4\,\mathrm{mg\,\cdot\,mg^{-1}\,\cdot\,d^{-1}}$\emph{.
}Because we fix the decomposition rate of given mass of prey, to maintain
a 48-hour decomposition time of a single $75-\mu g$ wasp while varying
\emph{a} and \emph{b} in Equation \ref{eq:decomp-negative exponential},
it is necessary to fix $\frac{a}{b}=5$. Finally, because \emph{S.
purpurea} produces digestive enzymes such as proteases and chitinases
only for a short period of time early in leaf growth \citep{Gallie1997},
we ignore their effects here.

Oxygen used up by the bacteria in prey consumption (= 1 - \textsc{bod})
was modeled as a saturating function of the remaining prey. The proportion
of the prey remaining = 1 \textendash{} the amount of prey decomposed,
which is a function of the maximum amount of oxygen at mid-day (\emph{a}
in Equation \ref{eq:1a:FullModel}, iteratively augmented by $a^{\prime}$from
Equation \ref{eq:1b:FullModelIterate} and as detailed below in Equations
\ref{eq:prey-mineralization}, \ref{eq:a-prime-augmentation}); the
mass of prey (\emph{w}); a half-saturation constant defining the prey-consumption
curve ($K_{w}$) that determines how much prey is left over each ``day''
of the iterated model; and the amount of oxygen needed by the bacteria
for basal metabolism. Hence:
\begin{equation}
\mathrm{O}_{2}\;\mathrm{lost}(t)=f(w,x,t)=m+a\left[\frac{w(t-1)}{K_{w}+w(t-1)}\right]\label{eq:O2lost-decomp}
\end{equation}


In the sensitivity analysis, \emph{m} in Equation \ref{eq:O2lost-decomp}
was fixed = 1, whereas \emph{K\textsubscript{w}} varied.


\section*{Diffusion and oxygenation}

{[}O\textsubscript{2}{]} increases in the pitcher fluid in three
ways. Some oxygen diffiuses from the atmosphere into the pitcher fluid,
but because this happens only at the surface of the pitcher fluid,
and the ``mouth'' of the pitcher is at least an order of magnitude
smaller than the surface area of the pitcher itself \citep{Ellison2002b};
we ignore this term and focus on re-oxygenation through baseline oxgyen
production and prey-augmented photosynthesis (Equation \ref{eq:O2 conditional}).
The latter results from a positive feedback loop in Equation \ref{eq:1b:FullModelIterate}
that links prey mineralization to the uptake of mineralized nutrients
by the plant, and the subsequent usage of these nutrients to increase
photosynthetic rate of the pitcher \citep{Farnsworth2008}. Ants and
wasps, the most common insect prey of pitcher plants \citep{Ellison2009},
are $\approx$50\% C, have a C:N ratio of 6:1, and N:P:K ratios of
$\approx$12:1.5:0.9 \citep{Sirota2013}. As the prey are mineralized,
the nutrients that are released, especially NH\textsubscript{4}\citep{Bradshaw1984}
and P (as \textsuperscript{32}P: \citealt{Plummer1964}), are absorbed
rapidly by the pitcher. Photosynthesis by \emph{Sarracenia} is limited
by both N and P (and stoichometrically by N: \citep{Ellison2006a,Wakefield2005}),
and photosynthetic rates of pitcher plants significantly increase
following N additions \citep{Ellison2002b}. We model these two processes
\textendash{} nutrient release and augmentation of photosynthesis
\textendash{} with a pair of equations.

First, nutrient release following bacterial mineralzation is a function
of prey mass (\emph{w}) and available oxygen (\emph{x}) used by bacteria
to break down and mineralize the prey:
\begin{equation}
n(t)=\frac{w(t)x(t)}{c}\label{eq:prey-mineralization}
\end{equation}


where \emph{c} is a scaling constant (we set \emph{c} = 100).

Nutrients absorbed by pitchers and not stored for future use \citep{Butler2007}
could be used to make additional enzymes for photosynthesis \citep{Givnish1984},
and we model uptake as a signmoidal (saturating) relationship betwen
additional nutrients absorbed and augmentation of the peak rate of
photosynthesis (\emph{a} in Equation \ref{eq:1a:FullModel}): 

\begin{subequations}
\begin{align}
a^{\prime}=\frac{a_{max}^{\prime}-a_{min}^{\prime}}{1+e^{-s \cdot n_t -d}}+a_{min}^{\prime}\label{eq:a-prime-augmentation}\\
a_{t+1}=a_{t}\,\times\,a^{\prime}_t\label{eq:a_{t+1}-update}
\end{align}


\end{subequations}

In Equation \ref{eq:a-prime-augmentation}, $a_{min}^{\prime}$ is
the minimum possible augmentation of photosynthesis, which we set
= 0; $a_{max}^{\prime}$is the maximum possible augmentation, which
we set = 2; \emph{s} is the steepness of the increase (set = 10);
and \emph{d} is the inflection point of the curve (= 0.5). Augmentation
evolves as the day's leftover prey (i.e., that not completely broken
down in a given day) accumulates (within the \emph{n}(\emph{t}) term),
and is mineralized on subsequent days. We note that modeling this
as discrete day-to-day carry-over is an artefact of iterations in
computer models. In real pitchers, decomposition is more continuous.
Similarly, the model updates the value of $a\prime$ once each ``day''
using Equation \ref{eq:a(t+1)-update} because conversion of nutrients
to new photosynthetic enzymes is assumed to occur slowly relatively
to bacterial decomposition itself.


\section*{Sensitivity analysis}

Five paramters were varied over a wide range of values to explore
the sensitivity of the model to prey input (mass added each day);
the decomposition rate (\emph{a} and \emph{b} in Equation \ref{eq:decomp-negative exponential}),
the half-saturation constant for prey consumption ($K_{w}$ in Equation
\ref{eq:O2lost-decomp}) and the inflection point \emph{d} for augmentation
of photosynthesis in Equation \ref{eq:a-prime-augmentation}. The
range of values for each of these parameters is given in Table \ref{tab:Sensitivity analysis parameter ranges}. 

\begin{table}
\centering{}\protect\caption{\label{tab:Sensitivity analysis parameter ranges}Values of parameters
used in the sensitivity analyses. Note that the values of \emph{a}
and \emph{b} must change in concert to maintain a constant ratio of
5 so that prey decomposition is complete within 48 hours}
\begin{tabular}{cc}
\hline 
Variable & values\tabularnewline
\hline 
Prey added ($w_{0})$ & 0, 1, 5 mg/day\tabularnewline
\emph{a} & 5, 10, 20, 40 mg\tabularnewline
\emph{b} & 1, 2, 4, 8 $\mathrm{mg\,\cdot\,mg^{-1}\,\cdot\,d^{-1}}$\tabularnewline
$K_{w}$ & 0.001, 0.01, 0.1 mg/min\tabularnewline
\emph{d} & 1, 2, 3, 4, 5 mg\tabularnewline
\hline 
\end{tabular}
\end{table}


Models were run for each of the possible combinations of $\{w_{0},\,a,\,b\,K_{w}\,d\}$\textemdash note
that the joint set of \emph{a} and \emph{b} includes only 4 combinations:
\{(5,1), (10,2), (20,4), (40,8)\}\textemdash for a total of 180 model
runs.


\part{Results}


\section*{Model dynamics}

Illustrative examples (?spark charts or some subset of them) illustrating
the range of possible outcomes - controls, intermediate dynamics,
collapse, recovery.

This should include time-series plots as well as phase-space plots.
Examples in Sirota et al. PNAS supplement, but I think we can do better/differently
here.
\begin{itemize}
\item Different state = change in O2 production where the non-transient
maximum values are different from control
\end{itemize}

\section*{Sensitivity analysis}

Which parameters drive the dynamics? Multivariate plots, additional
analyses, ...
\begin{itemize}
\item Transient = returning to the maximum
\item Prey: increases crashes
\item D: transient when D = -1, stable or crashing otherwise
\item K: increases the buffering of the system, increases return rates,
increases the number of stable states
\item ab: alters transient return rates, alters stable point
\end{itemize}

\part*{Discussion}

1. Why we need model systems for regime shifts and alternative states

2. Similarities and differences between pitcher plants, lakes, and
other ecological systems in which regime shifts have been identified
and explored

3. Insights from pitcher plant system and model that could be applied
to other systems


\section*{\pagebreak{}}


\part*{Literature Cited}

\bibliographystyle{amnatnat}
\bibliography{ecology}

\end{document}
